\subsection{Section Objective}

\textbf{Paragraph 1: Motivation and Potential of the Idea}

There are many ways to write this paragraph depending on your style.
One way is to describe the interesting potential of a new technology, technical capability, or set of algorithms. For example: 

\begin{quote}
    \SC{Learning from rare events, such as traffic disruptions or hazardous weather conditions, can improve the safety and reliability of autonomous vehicles (AVs). Today's AVs measure more than 4 terabytes (TB) of rich video and LIDAR sensory data in just a few hours, which can be mined to continually improve computer vision models \ldots}
\end{quote}

Focus on clear, short, simple sentences and cite survey papers if possible.

\textbf{Paragraph 2: Problems with State-of-the-Art Solutions}
Describe open technical challenges that need to be solved to achieve the goals and capabilities from Paragraph 1. Be precise, but avoid un-necessary technical jargon. Often, we will have a sentence saying \SC{``Despite the potential of [new technology], a key open problem is to [succinctly describe problem].''}

An example: 
\begin{quote}
    \SC{Despite the benefits of continually re-training vision models on large volumes of rich sensory data, we lack algorithms to balance these benefits with systems costs of network bandwidth consumption and cloud computing time.}
\end{quote}

\textbf{Paragraph 3: Key Technical Insight for Your Solution Approach}

Anyone can describe a grand challenge and open problems. Here, we describe your unique technical insight that allows you to solve the problem. First, describe \textbf{your key observation} on why state-of-the-art methods fail today. Then, describe your \textbf{key technical insight} that can provide a solution.
Then, describe the \textbf{the principal contribution} of your paper. 
Focus on clear, jargon-free sentences with carefully-selected technical terms so an expert can quickly appreciate what your solution and unique contribution will be. Often, reference Figure 1 that shows a clear diagram of your approach, but do not get into deep mathematical details. 

\subsubsection{Related Work}
First, describe the 4-5 key sub-topics your work relates to. For each, cite several papers and, crucially, describe how their approach \textit{differs} from yours or is complementary. Do not simply state what each paper does, but emphasize why it differs. Some people prefer to describe one closest competitor paper in detail: ``\SC{The closest work to ours is \cite{cheng2021data}. The key difference of our approach is we \ldots}''.

A \SC{good} example:

\begin{quote}
    \SC{Our work is broadly related to rate distortion theory, autoencoders, and task-driven representation learning in robotics. Several prior works use rate distortion theory to [brief overview and citations]. However, the standard assumption in rate distortion theory is to [describe why it differs], which is in stark contrast to our approach that [\ldots].}
\end{quote}

A \bad{bad} example:

\begin{quote}
\bad{
    Chinchali et. al. do [something]. Blank et. al. [do something else]. Continue with a long list of papers and their descriptions, but nothing to cluster them together based on sub-topic or contrast with your approach.} 
\end{quote}

Key words to use in this paragraph are \textbf{contrast, complementary} etc.

\subsubsection{Contributions }
``In light of prior work, our contributions are N-fold.'' Use a sentence like this to segway between related work and your key contributions. Make an enumerated list of contributions (typically 3-4) with clear, short sentences, links to code or data URLs, and quantitative numbers showing the improvement over benchmarks. 
Allude to cool hardware demonstrations or simulation platforms here.

\subsubsection{Paper Organization }
Describe the key content of each section and how they relate to the overall flow of the paper.

\subsection{Guidelines}
Outline the key contributions, technical insight, and comparison with state-of-the-art in \textbf{bullet points}. Then, only once you are done writing the rest of the paper, write the full text.

