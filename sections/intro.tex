\subsection{Section Objective and Outline}

\textbf{Paragraph 1: Motivation and Potential of the Idea}
There are many ways to write this paragraph depending on your style.
One way is to describe the interesting potential of a new technology, technical capability, or set of algorithms. For example: 

\begin{quote}
\SC{Learning from rare events, such as traffic disruptions or hazardous weather conditions, can improve the safety and reliability of autonomous vehicles. Today's AVs measure more than 4 TB of rich video and LIDAR sensory data in just a few hours, which can be mined to continually improve computer vision models \ldots}
\end{quote}

Focus on clear, short, simple sentences and cite survey papers if possible.

\textbf{Paragraph 2: Problems with State-of-the-Art Solutions}
Describe open technical challenges that need to be solved to achieve the goals and capabilities from Paragraph 1. Be precise, but avoid un-necessary technical jargon. Often, we will have a sentence saying \SC{``Despite the potential of [new technology], a key open problem is to [succinctly describe problem].''}

An example: 
\begin{quote}
    \SC{Despite the benefits of continually re-training vision models on large volumes of rich sensory data, we lack algorithms to balance these benefits with systems costs of network bandwidth consumption and cloud computing time.}
\end{quote}

\textbf{Paragraph 3: Key Technical Insight for Your Solution Approach}

\subsubsection{Related Work}

\subsubsection{Contributions }

\subsubsection{Paper Organization }

\subsection{Guidelines}
Outline the key contributions, technical insight, and comparison with state-of-the-art in \textbf{bullet points}. Then, only once you are done writing the rest of the paper, write the full text.

Dummy Citation \cite{alvar2019multi}
