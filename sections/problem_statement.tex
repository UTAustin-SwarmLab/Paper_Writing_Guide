\subsection{Section Objective}
The problem statement is the most important part of your paper and should be written first. First, allude to a diagram that shows the information flow with mathematical notation for your problem, such as Figure 2.
Then, systematically introduce key notation that helps you build up to a \textbf{formal mathematical optimization problem} with a cost function and constraints (in most problems our lab will focus on).

For a control or networking problem, describe the information flow. 
Namely, the sensory input, each computation function, and each function's input and output and task. Be formal, but provide a one sentence example for each.

\begin{enumerate}
    \item The sensory input.  Give the variable, an example,  and (during the first introduction) describe its dimension. ``\SC{The robot measures an n-dimensional sensory input $s_t \in \reals^n$, such as an image or LIDAR point cloud, at discrete time $t$.}''
    \item The controller's state space.
    \item The action space.
    \item The dynamics.
    \item The cost function.
\end{enumerate}

\subsubsection{Formal Problem Statement}
This should be in a formal problem statement block, such as in \cite{cheng2021data,nakanoya2021task}.
The statement should describe the inputs (``givens''), the optimization objective, and constraints. For the inputs and constraints, you likely will reference equations defined earlier. 

\subsubsection{Significance of the Problem}
Describe in one short paragraph why the problem is novel compared to state-of-the-art, how it broadly applies to many engineering settings, and what technical parts make it challenging to solve.

\subsection{Guidelines}

\subsubsection{Variables and Equations}

The following guidelines govern notation.

\begin{enumerate}
    \item \textbf{Time: } Use a subscript for \textit{discrete} time, such as $x_t$. A timeseries of variable $x$ from time $t_0$ to $t_f$ should be given by $x_{t_0:t_f}$.
    \item \textbf{Use Standard Notation: } In control, the state is generally given by $x_t$, control by $u_t$, policy by $\pi$, and cost function by $J$. Strictly follow the same notation as prior papers by Sandeep or key influential textbooks.
    \item \textbf{Intuitive Variable Choices: } Your reader has a limited attention span, and will forget random notation like $\psi, \Gamma$ unless it is necessary. Suppose we have $K$ epochs to run an algorithm. Instead of $K$, write $N_{\mathrm{epoch}}$, which is obvious in case I forget what $K$ is.
    \item \textbf{Deep Learning Notation: } Parameters of a model are given by $\theta$. If you have different models, call them $\theta_{\mathrm{enc.}}$, $\theta_{\mathrm{ctrl}}$ for an encoder and controller (for example), rather than use a slew of unintuitive varibles like $\alpha, \psi$ etc. Loss functions should be given by $\mathcal{L}$, datasets by $\mathcal{D}$, etc.
    \item Use $\textsc{mathrm}$ in LaTeX for English words in a LaTeX equation.
\end{enumerate}



