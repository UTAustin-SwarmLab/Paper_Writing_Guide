Each section is organized into two sub-sections.
The first, named ``Section Objective'', describes advice on what to place in each technical section. The second, named ``Guidelines'', enumerates good practices and common mistakes that should be avoided.

Before writing full text, make sure to \textbf{outline} your paper in bullet-points.
Example sentences are given in \SC{blue}.

\subsection{How to Maximally Benefit from This Guide}
Read each section and reference the corresponding section of our prior papers \cite{chinchali2021network,nakanoya2021task,ChinchaliSharmaEtAl2019}.
Try to map each guideline to what you see there. After you outline your paper, write Sections 2-N, one at a time. 
Focus on clear, short sentences and paragraphs. 

Before starting your research or writing, answer the following questions, based on DARPA's Heilmeier Catechism, in 1-2 sentences each. Put the questions and answers at the top of your outline.

\begin{enumerate}

\item What is the objective of your research? Minimize technical jargon.

\item If you achieve this goal, what benefit or new capability will it provide?

\item What are key failures with today's state-of-the-art?

\item What is your key technical insight that enables your solution?

\item What are the limitations and assumptions inherent to your approach?

\end{enumerate}
