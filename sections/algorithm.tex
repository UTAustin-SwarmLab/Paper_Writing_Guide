\subsection{Section Objective}
We have now formalized a challenging problem. Now, we want to introduce an algorithm or analytical solution. If we are doing pure deep learning, we should at least describe a loss function, model architecture, and training procedure. 

\subsubsection{Illustrative Toy Example}
Often, we describe a toy example that clearly demonstrates the idea works in simulation. First, define the toy problem parameters using the \textit{same notation} from the problem statement. Then, show the solution and a figure with the key results. Make sure to describe \textbf{the intuition} that the toy problem provides and how we can \textbf{generalize} the results to more sophisticated examples that will follow.

\subsubsection{Algorithm or Derivation}
This should artfully use insights from the toy problem to come up with a general algorithm or formal derivation. Tell why you do each step and place minute details for proofs in the Appendix if you have space constraints. 

A \bad{bad example} simply states \textbf{what} you did without logical transitions or stating \textbf{why}.

\subsubsection{Benchmarks}
This lists benchmarks that you will compare your algorithm with. In the list, put a formal, short name for the benchmark that will \textbf{uniformly} appear throughout the plots and rest of the text. Describe each benchmark in 2-3 sentences and argue why it is an appropriate benchmark, such as it is the current state-of-the-art, common practice, or tests a key feature of your algorithm.

\subsection{Guidelines}
An algorithm should appear as a stand-alone Figure with a caption. 
All algorithm lines should be \textbf{numbered} with \textbf{in-line comments} if space permits. Each line of the Algorithm should be described (and referenced) sequentially with intuition on why it matters. Do not simply state \textit{what} you do, but tell us \textit{why}. Here is an example: 

A \SC{good} example: 
\begin{quote}
    \SC{We now describe the key steps in Algorithm 1 to co-design a task-driven encoder and decoder. First, on line 1, we initialize the parameters $\theta$ randomly. Then, our next step is to $\ldots$.}
\end{quote}
