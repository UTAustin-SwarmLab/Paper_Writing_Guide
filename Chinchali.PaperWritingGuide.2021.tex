\documentclass[conference]{IEEEtran}
\usepackage{times}

% numbers option provides compact numerical references in the text. 
\usepackage[numbers]{natbib}
\usepackage{multicol}
\usepackage[bookmarks=true]{hyperref}

\usepackage{amsmath,amsthm}

\pdfinfo{
   /Author (Homer Simpson)
   /Title  (Robots: Our new overlords)
   /CreationDate (D:20101201120000)
   /Subject (Robots)
   /Keywords (Robots;Overlords)
}

%\sloppy
%\frenchspacing
%\renewcommand{\baselinestretch}{1.0}

\usepackage{color}
\newcommand{\SC}[1]{{\color{blue}#1}}



%%% ------------------------------------------------------------------------------------------ %%%
\title{\LARGE \bf How to Write Technical Papers: UT Austin Swarm Lab}

\author{\IEEEauthorblockN{
Sandeep Chinchali\IEEEauthorrefmark{1}
\IEEEauthorblockA{Department of Electrical and Computer Engineering\IEEEauthorrefmark{1},
The University of Texas at Austin, Austin, TX}
\IEEEauthorblockA{E-mail: sandeepc@utexas.edu}}}

\begin{document}
\maketitle

\begin{abstract}
This document provides guidelines on how to structure a technical paper. 
My hope is that by following these key steps, you can save a significant amount of time
during your PhD. 

\end{abstract}

\IEEEpeerreviewmaketitle

%%%%%%%%%%%%%%%%%%%%%%%%%%%%%%%%%%%%%%%%%%%%%%%%%%%%%%%%%%%%%%%%
\section{Introduction}
\label{sec:introduction}
Intro

Dummy Citation \cite{alvar2019multi}


\section{Problem Statement}
\label{sec:problem_statement}
\subsection{Objective}
The problem statement is the most important part of your paper and should be written first. First, allude to a diagram that shows the information flow with mathematical notation for your problem, such as Figure 2.
Then, systematically introduce key notation that helps you build up to a \textbf{formal mathematical optimization problem} with a cost function and constraints (in most problems our lab will focus on).

For a control or networking problem, describe the information flow. 
Namely, the sensory input, each computation function, and each function's input and output and task. Be formal, but provide a one sentence example for each.

\begin{enumerate}
    \item The sensory input.  Give the variable, an example,  and (during the first introduction) describe its dimension. ``\SC{The robot measures an n-dimensional sensory input $s_t \in \reals^n$, such as an image or LIDAR point cloud, at discrete time $t$.}''
    \item The controller's state space.
    \item The action space.
    \item The dynamics.
    \item The cost function.
\end{enumerate}

\subsection{Variables and Equations}

The following guidelines govern notation.

\begin{enumerate}
    \item \textbf{Time: } Use a subscript for \textit{discrete} time, such as $x_t$. A timeseries of variable $x$ from time $t_0$ to $t_f$ should be given by $x_{t_0:t_f}$.
    \item \textbf{Use Standard Notation: } In control, the state is generally given by $x_t$, control by $u_t$, policy by $\pi$, and cost function by $J$. Strictly follow the same notation as prior papers by Sandeep or key influential textbooks.
    \item \textbf{Intuitive Variable Choices: } Your reader has a limited attention span, and will forget random notation like $\psi, \Gamma$ unless it is necessary. Suppose we have $K$ epochs to run an algorithm. Instead of $K$, write $N_{\mathrm{epoch}}$, which is obvious in case I forget what $K$ is.
    \item \textbf{Deep Learning Notation: } Parameters of a model are given by $\theta$. If you have different models, call them $\theta_{\mathrm{enc.}}$, $\theta_{\mathrm{ctrl}}$ for an encoder and controller (for example), rather than use a slew of unintuitive varibles like $\alpha, \psi$ etc. Loss functions should be given by $\mathcal{L}$, datasets by $\mathcal{D}$, etc.
    \item Use $\textsc{mathrm}$ in LaTeX for English words in a LaTeX equation.
\end{enumerate}



\section{Solution Approach}
\label{sec:task_specific_algorithm}
\subsection{Section Objective}
We have now formalized a challenging problem. Now, we want to introduce an algorithm or analytical solution. If we are doing pure deep learning, we should at least describe a loss function, model architecture, and training procedure. 

\subsubsection{Illustrative Toy Example}
Often, we describe a toy example that clearly demonstrates the idea works in simulation. First, define the toy problem parameters using the \textit{same notation} from the problem statement. Then, show the solution and a figure with the key results. Make sure to describe \textbf{the intuition} that the toy problem provides and how we can \textbf{generalize} the results to more sophisticated examples that will follow.

\subsubsection{Algorithm or Derivation}
This should artfully use insights from the toy problem to come up with a general algorithm or formal derivation. Tell why you do each step and place minute details for proofs in the Appendix if you have space constraints. 

A \bad{bad example} simply states \textbf{what} you did without logical transitions or stating \textbf{why}.

\subsubsection{Benchmarks}
This lists benchmarks that you will compare your algorithm with. In the list, put a formal, short name for the benchmark that will \textbf{uniformly} appear throughout the plots and rest of the text. Describe each benchmark in 2-3 sentences and argue why it is an appropriate benchmark, such as it is the current state-of-the-art, common practice, or tests a key feature of your algorithm.

\subsection{Guidelines}
An algorithm should appear as a stand-alone Figure with a caption. 
All algorithm lines should be \textbf{numbered} with \textbf{in-line comments} if space permits. Each line of the Algorithm should be described (and referenced) sequentially with intuition on why it matters. Do not simply state \textit{what} you do, but tell us \textit{why}. Here is an example: 

A \SC{good} example: 
\begin{quote}
    \SC{We now describe the key steps in Algorithm 1 to co-design a task-driven encoder and decoder. First, on line 1, we initialize the parameters $\theta$ randomly. Then, our next step is to $\ldots$.}
\end{quote}


\section{Experimental Results}
\label{sec:experiments}
\subsection{Section Objective}
We have now described our problem and what we \textit{claim} is the best solution.
We now need to provide hard quantitative evidence that proves this is true on diverse datasets or experimental domains. For a fair comparison, we need to describe: 

\begin{enumerate}
    \item \textbf{Evaluation Metrics: } What metrics define a good solution to the optimization problem stated in the problem formulation? Typically, this is a low overall cost or loss, but we also should describe individual terms of the loss function. These metrics \textbf{must appear} in a table or clear figure for \textbf{all benchmarks}. 
    \item \textbf{Benchmarks: } Recap the benchmarks and specific variants of your algorithm you will test. 
    \item \textbf{Diverse Experimental Domains: } Allude to each dataset you test your algorithm on in order of sophistication. Ideally, the last experiment should be on a real robot or hardware.
\end{enumerate}

\subsection{Guidelines}

\subsubsection{Figure and Caption Guidelines}

All figures should be described and referenced in order in the text. 
Each caption should be descriptive and have a short title in \textbf{bold} describing the key take-away. The caption should describe why your method is better than benchmarks etc. Never have short, obvious captions that describe the plot title, such as ``Plot of Accuracy vs. Latency''.

\subsubsection{Plots}

The plots should be done using a combination of Matplotlib and the Seaborn Packages in python. All raw data to create the plot should be stored and backed up in GIT as a csv. Axes tick marks, legends, and axes labels should be \textit{at least} \textbf{20 point font}.

If your axes titles or legends describe a mathematical quantity, first state the English name and the variable in LaTeX. For example, \textbf{Acceleration $a$}.

The following guidelines are for lines/curves:

\begin{enumerate}
    \item Line width/thickness \textbf{3} or larger.
    \item Different color \textbf{and} line style (dashed, dots, etc.) for people who print out the paper in black and white.
    \item Use a descriptive legend.
\end{enumerate}



\section{Discussion and Conclusions}
\label{sec:conclusion}
\subsection{Section Objective}

\textbf{Paragraph 1: Re-cap of novelty and key results}

Highlight the key technical insight, contributions, and key numbers that describe why your method works so well compared to benchmarks. This should be similar to the contributions, but stress intuition and key novelty/insights. 

\textbf{Paragraph 2: Future Work}

Describe next steps, open research questions, and why this line of work is important for the research community. Remind readers if the code or dataset is open-source and link to the URL. If you have read this far and are Sandeep's student (or student collaborator), send him an email with today's date and describe something you have learned from this document.
Then, and only then, start outlining your paper. 



\section{Acknowledgements}

%\newpage
%\bibliographystyle{plainnat}
\bibliographystyle{abbrv}
\bibliography{ref/full}

\newpage
\appendix
\label{sec:appendix}
The appendix should have detailed experimental settings, model architectures and hyper-parameters for deep learning experiments, and links to all URLs for datasets. 

The appendix should also have the \textbf{full proof} for any theorems you cannot fit into the main document.

\end{document}
